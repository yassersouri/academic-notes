\documentclass{report}
\usepackage{graphicx}
\usepackage{xepersian}
\usepackage{geometry}
\settextfont[Scale=1.2]{XB Zar}
\renewcommand{\baselinestretch}{1.8}

% absolute position title
\usepackage{textpos}

% section numbering
\renewcommand{\thesection}{\arabic{section}}
\renewcommand{\thesubsection}{\thesection.\arabic{subsection}}
\renewcommand{\thesubsubsection}{\thesection.\arabic{subsection}.\arabic{subsubsection}}

\title{
\begin{normalsize}
به نام خدا
\end{normalsize}
\\[2cm]
بررسی مقاله
\\[1cm]
واسنجی مقاوم دوربین برای ویدئوهای ورزشی با استفاده از مدل زمین ورزشی
}
\author{یاسر سوری
\\
\\ \small دانشگاه صنعتی شریف
\\ \small souri@ce.sharif.edu
}
\begin{document}
\maketitle
%\begin{textblock*}{15cm}(0cm,-8cm)\centering به نام خدا \end{textblock*}

\begin{abstract}
در این مقاله روشی برای بدست آوردن ماتریس هوموگرافی بین زمین و تصویر دوربین، برای تصاویر ورزشی که مدل زمین آن‌ها را می‌دانیم بیان شده است. مانند بسیاری از مقالات دیگر این مقاله یک مرحله مقدار دهی اولیه برای پارامترهای دوربین دارد و بعد از آن برای هر فریم جدید، دوربین را دنبال  می‌کند\LTRfootnote{tracking}. همچنین فرض شده است که مدل زمین شامل حداقل دو خط افقی موازی و دو خط عمودی موازی است. برای همین الگوریتم برای تصاویر مرکز زمین فوتبال پاسخ‌گو نیست. از دیگر فرضیات مهم این مقاله ثابت بودن مکان و \lr{lens distortion} دوربین در طول گذر زمان است.
\\
لازم به ذکر است که نویسنده این مقاله، در سال بعد ورژن بعدی راه حلش را در مقاله‌ای دیگری چاپ کرده است\cite{new_paper}. 

\end{abstract}

\section{فرض‌های مسئله}
در این بخش بعضی از فرضیات مسئله‌ای که قرار است در این مقاله حل شود مورد بررسی قرار می‌گیرد.
\begin{itemize}
\item
فرض شده است که، پارامتر‌های هشتگانه ماتریس هوموگرافی بین زمین و تصویر دوربین را بدست می‌آوریم، ولی پارامترهای \lr{pan}، \lr{tilt} و \lr{zoom} به صورت مجزا محاسبه نمی‌شوند.
\item
همچنین، پارامترهای مکان دوربین، \lr{roll} و \lr{lens distortion} در طول زمان تغییر نمی‌کنند.
\item
\end{itemize}
\subsection{ایده گرفتن از مقالات دیگر}
در این مقاله از دو مقاله‌ی دیگر ایده گرفته شده است و بهبود‌هایی روی آن‌ها انجام شده است. این مسئله بسیار برای من جالب بود. در ادامه این ایده‌گیری را توضیح خواهم داد.

در این مقاله دو بخش اصلی وجود دارد. اول مقدار دهی اولیه \LTRfootnote{initialization} که از ایده‌ی مقاله \cite{ct16} استفاده شده است. دوم بخش دنبال کردن است که از ایده‌ی مقاله \cite{ct15} استفاده شده است. در بخش مقدار دهی اولیه و آغاز فرآیند دنبال کردن در \cite{ct16} از یک روش جستجوی خسته کننده \LTRfootnote{exhaustive} استفاده کرده است. این جستجو کند بوده و برای عملکرد بلادرنگ مناسب نمی‌باشد، در این مقاله بهبودی روی آن داده شده است که سرعت آن را تا ۱ ثانیه بالا می‌برد. در بخش دنبال کردن مقاله‌ی \cite{ct15} روشی بر اساس دنبال کردن خط مطرح شده است که در این مقاله تشخیص خط آن بهبود داده شده است.
این نکته که ایده‌ی این مقاله کاملا در چند مقاله‌ی دیگر بوده است برای بنده جالب بود. در این مقاله عملا با ایده‌های خوب و به جا، توانسته یک سیستم مناسب برای واسنجی دوربین ارائه کند که توانسته در دو سامانه‌ی تجاری مورد استفاده قرار بگیرد.
\section{کلیت روش}
این مقاله دارای دو الگوریتم متفاوت است:
\begin{itemize}
\item
جستجو: خروجی این قسمت، پیدا کردن ماتریس دوران و همچنین فاصله‌ی کانونی است.
\item
دنبال کردن: این قسمت یک مسئله‌ی بهینه سازی است که می‌توان، آن را برای پیدا کردن تمام پارامترها یا بخشی از آن‌ها حل کرد.
\end{itemize}
همچنین این مقاله دارای سه قسمت اصلی است:
\begin{itemize}
\item
پیدا کردن پارامترهای ثابت، یعنی مختصات دوربین و همچنین شیب زمین.
\item
مقداردهی اولیه برای پارامترهایی که تغییر می‌کنند، یعنی ماتریس دوران و فاصله‌ی کانونی.
\item
دنبال کردن پارامترهایی که تغییر می‌کنند.
\end{itemize}
لازم به ذکر است که در قسمت اول و سوم از الگوریتم دنبال کردن و در قسمت دوم از الگوریتم جستجو استفاده می‌شود.

همچنین در این مقاله روش خوبی برای تشخیص خط بیان شده است که ادعا شده خوب عمل می‌کند.
\subsection{مدل زمین فوتبال}
مدل زمین فوتبال شامل تعدادی خط است. حتی قسمت‌های دایر‌ه‌ای خطوط زمین هم به صورت خط صاف در نظر گرفته شده‌اند. به این صورت که هر ۱۰ درجه از دایره‌ها را یک خط صاف در نظر می‌گیریم. توجه کنید که ابعاد زمین فوتبال را به طور دقیق محاسبه کرده‌ و می‌دانیم. به همین خاطر هنگامی که پارامترهای دوربین را داشته باشیم می‌توانیم مدل زمین را بر روی تصاویر دوربین رسم کنیم. از این رسم کردن (انداختن) مدل زمین روی تصاویر در مرحله‌ی جستجو استفاده می‌کنیم.
\subsection{جستجو}
در الگوریتم جستجو ما فرض می‌کنیم که ما پارامترهای ثابت دوربین‌ها را داریم. سپس برای یک تصویر پارامترهایی که تغییر می‌کنند را (ماتریس دوران و فاصله‌ی کانونی) محاسبه می‌کنیم. این محاسبه می‌تواند به خودی خود برای دوربین‌های ثابت مورد استفاده قرار بگیرد یا برای مقدار دهی اولیه در عملیات دنبال‌کردن دوربین‌های مورد استفاده قرار گیرد. از این به بعد فرض می‌کنیم که فاز جستجو برای مقدار دهی اولیه مورد استفاده قرار می‌گیرد.

برای مقدار دهی اولیه زاویه‌های \lr{pan} و \lr{tilt} و مقدار فاصله‌ی کانونی برای یک تصویر خاص، ابتدا فرض می‌کنیم که از تصویر ورودی را تبدیل به تصویر باینری کرده‌ایم که در آن نقاطی که ممکن است روی خطوط قرار داشته باشند در آن مشخص شده است. مرحله‌ی تشخیص خطوط از تبدیل هاف\LTRfootnote{Hough transform} استفاده می‌کند با این تفاوت که تصویر را به ۱۰ قسمت مساوی (عمودی یا افقی، بسته به نزدیک به عمودی یا نزدیک به افقی بودن خطوط زمین) تقسیم می‌کند و برای هر کدام از آن قسمت‌ها یک تبدیل هاف جدا را محاسبه می‌کند. لازم است که توجه کنید برای هر بار مقدار دهی اولیه، یک بار تبدیل هاف تصویر حساب می‌شود.

سپس برای مقادیر مختلف ممکن از پارامترهای \lr{pan}، \lr{tilt} و فاصله‌ی کانونی با اندازه‌ی گام مشخصی، مدل زمین را بر روی تصویر می‌اندازیم. حال برای هر خط از خطوط مدل زمین که در تصویر قابل رؤیت است، بسته به اینکه در کدام نواحی ۱۰ گانه در تقسیم بندی قرار دارد، نقطه‌ی معادل را در فضاهای هاف ده‌گانه پیدا می‌کنیم و مقدار آن نقطات را در فضای هاف برای تمام خطوط زمین با یکدیگر جمع می‌کنیم. حاصل این جمع به ما ارزش این پارامترهای دوربین را می‌دهد.

در انتهای جستجو از بین تمام پارامترهای مختلف دوربین، آن ۱۰ تایی را انتخاب می‌کنیم که دارای ارزش بیش‌تری هستند. در مراحل دنبال کردن دوربین در حقیقت ۱۰ دنبال کردن همزمان را انجام می‌دهیم. ولی پس از چند مرحله از دنبال کردن مشخص می‌شود که فقط یکی از این مقادیر خوب دنبال می‌شود و آن را انتخاب می‌کنیم. همچنین باید توجه کنیم که در انتخاب ۱۰ تا از سه‌تایی پارامترهای \lr{pan}، \lr{tilt} و فاصله‌ی کانونی که دارای بیش‌ترین ارزش هستند، از انتخاب مقادیر نزدیک به هم که عملا ماکزیمم‌های محلی هستند اجتناب می‌کنیم و آن‌هایی را انتخاب می‌کنیم که دارای تفاوت حداقلی در زوایا باشند.
\subsection{دنبال کردن}
در مرحله‌ی دنبال کردن، ما یک حدس اولیه از پارامترهایی دوربین که همان مقادیر زمان (فریم) قبل هستند را داریم و حال می‌خواهیم برای فریم جدیدی که دوربین تصویر آن را ضبط کرده است، پارامترهای جدید را بدست آوریم. این مرحله در حقیقت یک روش بهینه‌سازی گام به گام \LTRfootnote{iterative} است که می‌توان آن را برای پارامترهای ثابت، متحرک یا هر دو دسته پارامترهای دوربین حل کرد. برای اطلاع از جزئیات بیش‌تر این قسمت به خود مقاله مراجعه کنید.
\section{موارد مبهم}
این مقاله دو قسمت مبهم برای من دارد:
\begin{itemize}
\item
اول در مرحله‌ی تعیین شیب زمین و موقعیت دوربین، چگونه مسئله‌ی بهینه سازی را حل می‌کند؟
\item
دوم اینکه در مرحله‌ی جستجو، گام را به چه طریقی تعیین کرده است؟
\end{itemize}
\section{پیاده‌سازی}
به نظر می‌رسد که این مقاله و راه حلی که ارائه داده است، قابلیت حل کردن مسئله‌ی ما را داشته باشد. از آنجایی که در صورت مسئله‌ی ما، موقعیت دوربین‌ها تعیین شده است، در فازهای اول پیاده‌سازی این مقاله نیازی به انجام دادن این کار نیست.

ضمنا اکثر دوربین‌های ما ثابت هستند. این مسئله هم یک حسن است و هم یک عیب. حسن از این نظر که برای واسنجی این دوربین‌ها نیازی به پیاده‌سازی مرحله‌ی دنبال کردن نداریم. و عیب از این نظر که در روش این مقاله خروجی مرحله‌ی جستجو ۱۰ سری پارامتر برای دوربین است که در مرحله‌ی دنبال کردن بهترین آن‌ها تعیین می‌شود که از آنجایی که دوربین ما تکان نمی‌خورد، ما نمی‌توانیم بهترین سری پارامترها را با روش این مقاله انتخاب کنیم. پس باید فکری به حال انتخاب بهترین سری پارامترها بکنیم.

\renewcommand{\baselinestretch}{1}
\renewcommand*{\refname}{\section{منابع}}

\begin{latin}
\bibliography{court_model_bib}{}
\bibliographystyle{plain}

\end{latin}

\end{document}