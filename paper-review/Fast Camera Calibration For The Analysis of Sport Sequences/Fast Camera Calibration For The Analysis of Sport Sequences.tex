\documentclass{report}
\usepackage{graphicx}
\usepackage{xepersian}
\usepackage{geometry}
\settextfont[Scale=1.2]{XB Zar}
\renewcommand{\baselinestretch}{1.8}


% absolute position title
\usepackage{textpos}

% section numbering
\renewcommand{\thesection}{\arabic{section}}
\renewcommand{\thesubsection}{\thesection.\arabic{subsection}}
\renewcommand{\thesubsubsection}{\thesection.\arabic{subsection}.\arabic{subsubsection}}

\title{
\begin{normalsize}
به نام خدا
\end{normalsize}
\\[2cm]
بررسی مقاله
\\[1cm]
واسنجی سریع دوربین برای تحلیل ویدئوهای ورزشی
}
\author{یاسر سوری
\\
\\ \small دانشگاه صنعتی شریف
\\ \small souri@ce.sharif.edu
}
\begin{document}
\maketitle
%\begin{textblock*}{15cm}(0cm,-8cm)\centering به نام خدا \end{textblock*}

\begin{abstract}
این مقاله\cite{new_paper} نسخه‌ی سریع‌تر مقاله‌ی قبلی\cite{old_paper} توسط همین نویسنده است. در این مقاله سعی شده است که سرعت واسنجی تا حد بلادرنگ افزایش یابد. کلیت الگوریتم نسبت به نسخه‌ی قبلی حفظ شده است، ولی اجزای مختلف با اجزایی با کارایی بهتر جایگزین شده است. فرضیات این مقاله مثل فرضیات مقاله‌ی قبلی است و از این نظر با هم تفاوتی ندارند.

\end{abstract}

\section{فرض‌های مسئله}
در این بخش بعضی از فرضیات مسئله‌ای که قرار است در این مقاله حل شود مورد بررسی قرار می‌گیرد.
\begin{itemize}
\item
فرض شده است که، پارامتر‌های هشتگانه ماتریس هوموگرافی بین زمین و تصویر دوربین را بدست می‌آوریم، و پارامترهای \lr{pan}، \lr{tilt} و \lr{zoom} به صورت مجزا محاسبه نمی‌شوند.
\item
همچنین، پارامترهای مکان دوربین، \lr{roll} و \lr{lens distortion} در طول زمان تغییر نمی‌کنند.
\item
مدل زمین ورزشی (شامل خط‌های زمین، طول و فاصله‌ی آن‌ها از هم) را می‌دانیم.
\end{itemize}
\subsection{نکات}
\begin{itemize}
\item

\end{itemize}
\section{کلیت روش}
این مقاله دارای دو الگوریتم متفاوت است:
\begin{itemize}
\item
\textbf{مقداردهی اولیه‌ی پارامترهای دوربین}: خروجی این قسمت ماتریس هوموگرافی است. در مقاله بیان شده است که حدود ۱ ثانیه هم به زمان نیاز دارد.
\item
\textbf{دنبال کردن}: در این قسمت، مارتیس هوموگرافی کنونی و لحظه‌ی قبل را استفاده می‌کند و حدسی در مورد ماتریس هوموگرافی لحظه‌ی بعد می‌زند. سپس با استفاده از یک مرحله سریع بهینه‌سازی این حدس را بهبود می‌بخشد. در مقاله بیان شده است که این الگوریتم سریع است و به صورت بلادرنگ می‌تواند اجرا شود.
\end{itemize}
همچنین این مقاله دارای چهار قسمت اصلی است:
\begin{itemize}
\item
پیدا کردن پیکسل‌های سفید که احتمالا بر روی خطوط مدل زمین قرار دارند.
\item
استخراج‌های خطوط مستقیم کاندید برای خطوط مدل زمین.
\item
پیدا کردن سازگاری مناسب بین خطوط کاندید و خطوط مدل. این مرحله بیش‌ترین زمان را می‌برد.
\item
بهبود پارامترهای دوربین.
\end{itemize}

برای هر دنباله‌ای از تصاویر مانند شکل  ابتدا در فریم اول، مقداردهی اولیه پارامترهای دوربین (الگوریتم اول) انجام می‌شود که شامل هر چهار قسمت نام برده شده در بالاست. سپس در فریم‌های بعدی الگوریتم دنبال کردن استفاده می‌شود که شامل مرحله‌ی پیدا کردن پیکسل‌های سفید روی خط و بهبود پارامترهای دوربین است.

\subsection{مدل زمین}
فرض شده است که زمین‌های ورزشی، به صورت صفحه هستند. به همین خاطر نگاشت بین مختصات دنیای واقعی و مختصات تصویر یک نگاشت هوموگرافی\LTRfootnote{Homography} است. برای پیدا کردن پارامترهای این ماتریس نیاز به حداقل ۴ نقطه‌ی متناظر داریم. در این مقاله از نقاط تلاقی خطوط زمین به عنوان این نقاط متناظر استفاده شده است.

برای حداقل ۴ نقطه‌ی منتاظر در تصویر و در مدل زمین، باید ۲ خط عمودی در مدل زمین و ۲ خط افقی را در آن بیابیم و منتاظر این خطوط را نیز در زمین پیدا کنیم. حاصل این ۲ خط عمودی و ۲ خط افقی، ۴ نقطه‌ی تلاقی خواهد بود که مختصاتشان را در تصویر و در دنیای واقعی داریم.

\begin{latin}
\bibliography{court_model_bib}{}
\bibliographystyle{plain}
\end{latin}

\end{document}